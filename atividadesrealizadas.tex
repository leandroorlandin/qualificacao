Este capítulo apresenta um estudo exploratório realizado com o objetivo de verificar a viabilidade e aperfeiçoar a proposta com base nas descobertas e desafios enfrentados durante o processo. Destaca-se que este estudo teve o foco apenas na análise das avaliações de acessibilidade e não explorou as sugestões de modificações e as alterações realizadas no código das aplicações. Todas as decisões de projeto e metodologia adotada estão em conformidade com o que foi descrito no Capítulo~\ref{chap:proposta}.
Além disso, também são descritas as atividades já realizadas do cronograma proposto.

\section{Questões de pesquisa}

As seguintes questões de pesquisa foram definidas para este estudo exploratório:
\begin{itemize}
 \item RQ1 - Quantas avaliações de usuário são relacionadas a acessibilidade e qual é a sua distribuição? 
 \item RQ2 - Qual é a diversidade de tópicos de acessibilidade abordados nas avaliações dos usuários?
 \item RQ3 - Quais são as notas associadas às avaliações que abordam aspectos de acessibilidade?  
\end{itemize}

\section{Seleção de aplicativos móveis}

Os critérios de seleção dos aplicativos para este estudo seguem os critérios definidos na Seção~\ref{sec:selecao}. Portanto, foram selecionadas aplicações da plataforma Android que satisfaziam as seguintes condições: estavam disponíveis na Google Play Store e o código-fonte estava armazenado em um repositório público da plataforma GitHub. Ao todo, apenas 701 aplicativos dentre os mais de 2000  indexados no FDroid\footnote{Dados de julho/2019} satisfizeram esses critérios.

A Tabela~\ref{tab:summarygps} mostra a distribuição de alguns atributos dos aplicativos selecionados por meio de estatística descritiva utilizando o resumo dos cinco números (mínimo amostral, quartil inferior, 
mediana, quartil superior, máximo amostral), a média e o desvio padrão. 
A maioria dos aplicativos tem até 9 atividades, mas há aplicativos muito maiores, como o ``Slide for Reddit'', por exemplo, que possui 91 atividades.

\begin{table}[htb]
%\setlength{\tabcolsep}{0pt}

\centering
\caption{Statistics on the Google Play Store sample}
\small
\label{tab:summarygps}
\begin{tabular}{lrrrrr}
\hline
             & Atividades & Nota & Avaliações     & Instalações  & Comentários \\
\hline
Mínimo          & 1          & 0     & 0            & 0         & 0       \\
Quartil Inferior           & 2          & 4           & 50    & 1K      & 4       \\
Mediana       & 5          & 4.3   & 130         & 10K     & 22      \\
Quartil Superior          & 9          & 4.6   & 836        & 50K     & 145     \\
Máximo          & 92         & 5     & 3.6M    & 100M & 4480    \\
Média         & 8.16       & 4.16  & 12.4K     & 550K+    & 305.4   \\
Desvio Padrão        & 10.5       & 0.79  & 151.6K  & 5.7M   & 840.8  \\
\hline
\end{tabular}
\end{table}

A maioria dos aplicativos possui uma boa nota e a maioria foi avaliada por até 836 usuários. O número de avaliações apresentado na tabela é diferente do número de comentários, pois quando um usuário faz uma avaliação ele é obrigado a atribuir uma nota, mas não é obrigado a inserir um comentário. Existem aplicativos que possuem um número muito alto de avaliações, como o Telegram, que possui 3,6 milhões de avaliações. O Telegram também é o aplicativo mais instalado (100 milhoes de istalações).
Por outro lado, o número de comentários não é muito alto, pois a maioria dos aplicativos recebeu até no máxmio 145 comentários em suas avaliações. O número máximo de revisões desta amostra é de 4480 em razão das limitações da API utilizada. Esta limitação não teve muito impacto neste estudo exploratório porque apenas 15 aplicativos (2\%) possuem mais de 4480 revisões registradas na Google Play Store.


\section{Extração e seleção das avaliações}

Uma aplicação escrita na linguagem Python foi escrita para consumir os dados da Google Play Store utilizando a API \emph{google-play-api}\footnote{https://github.com/facundoolano/google-play-api}. Como mencionado anteriormente, na versão utilizada, esta API só permite o retorno de no máximo 4480 avaliações de cada aplicativo. Além disso, só foram recuperadas avaliações escritas em inglês, que é o idioma padrão definido pela API.


A seleção das avaliações que possivelmente estão relacionadas a algum aspecto de acessibilidade da aplicação foi feita por meio da utilização de palavras-chave aplicadas aos comentários dos usuários. 
Para isso, um conjunto de palavras-chave foi definido com base nas diretrizes de acessibilidade da BBC~\cite{bbc}. Este guia de acessibilidade possui 54 recomendações que são classificadas em onze diferentes categorias: áudio e vídeo (5); design (12); editorial (3); foco (6); formulários (6); imagens (2); links (3); notificações (4); scripts e conteúdo dinâmico (4); estrutura (4); e texto equivalente (5). 
Este padrão foi selecionado porque ele tem o foco em aplicações móveis e possui exemplos de implementação e de teste de cada recomendação, o que permite compreender melhor os tipos de problemas de acessibilidade tratados no guia. 

Foram definidas 213 palavras-chave com base na análise de cada recomendação do guia. 
A Tabela~{tab:keywords} mostra exemplos das palalavras-chave utilizadas.
Note que foram utilizadas variantes das palavras (exemplo: \textit{cannot see} e \textit{can't see}, ou \textit{color} e \textit{colour}, ou \textit{impaired} e \textit{impairement}) para garantir que nenhuma avaliação relevante fosse excluída. Infelizmente, não é possível capturar casos em que as palavras foram escritas com a grafia errada. A lista completa das palavras-chave podem ser vistas em um repositório criado para armazenar dos dados deste estudo\footnote{https://github.com/marceloeler/data-ihc2019}.


\begin{table}[!htb]
\small
\caption{Examples of the keywords used to identify user reviews that may be related to accessibility}
\label{tab:keywords}
\begin{tabular}{|l|l|}
\hline
Categorias & Palavras-chave \\
\hline
Gerais                    & accessibility, disability, screen reader, Talkback, operable, impaired, 
\\& impairment                                               \\

\hline
Áudio/Vídeo             & subtitle, sign language, audio description,
 transcript, autoplay, mute, \\& volume                                                 \\

\hline
Design                      & contrast, background color, blind, flicker,
 visual cue, touch size, \\&overlap, font size, 
 dark/light mode, eyestrain, seizure, can't see \\

\hline
Editorial                   & consist. label, language, visual/audio cue                                                                \\

\hline
Foco                       & focusable, control focus, keyboard trap, 
 focus order, navigable                        \\

\hline
Formulários                       & unique label, missing label, content description,
 input type, \\& input format, focusable                                                                        \\

\hline
Imagens                      & image of text, hidden text, text alternative,
background image                                                                                               \\

\hline
Links                       & link description, unique desc., duplicate
link, alternative format \\

\hline
Notificações               & inclusive, haptic, vibration, feedback, alert 
 dialog, understandable, \\& unfamiliar                                                                             \\

\hline
Conteúdo dinâmico & animated content, page refresh, automatic  
refresh, timeout,  adaptable, \\&input sign                                                               \\

\hline
Estrutura                   & page title, screen title, heading, header                                                                                                         \\
\hline
Texto equivalente             & alternative text, non-visual, blind, screen reader, content description  \\
\hline
\end{tabular}
\end{table}

O uso de palavras-chave pode trazer muitos falsos positivos, uma vez que várias palavras utilizadas podem ter conotações diferentes em diferentes contextos. Por isso foi realizada uma análise manual de todas as avaliações que possuíam pelo menos uma palavra-chave definida. Neste estudo, apenas uma pessoa fez a análise manual das avaliações. Além disso, as avaliações não selecionadas por palavras-chave não foram analisadas para saber se alguma delas não foi selecionada pela ausência de alguma palavra-chave relevante e que não foi utilizada. Durante a análise manual, as avaliações confirmadas também foram classificadas em: 
requisições, quando o usuário reporta algum problema de acessibilidade ou solicita alguma modificação ou adição para tornar a aplicação mais acessível;
ou elogios, quando o usuário parabeniza o aplicativo pelo seu nível de acessibilidade.

Neste estudo, foram extraídas 214.053 avaliações dos 701 aplicativos da amostra. 
Dessas avaliações, apenas 5.076 foram pré-selecionadas por meio das palavras-chave. 
Após a análise manual, o número de avaliações de fato relacionadas à acessibilidade das aplicações foi de 2.663. 

