
Neste capítulo são apresentados alguns conceitos fundamentais de acessibilidade, bem como a descrição dos padrões de acessibilidade utilizados como referência para a condução dos estudos apresentados nesta proposta de pesquisa. Além disso, é apresentada uma visão geral dos trabalhos relacionados.

\section{Conceitos fundamentais}

Acessibilidade digital refere-se à capacidade de um produto ou conteúdo digital de ser percebido, compreendido e operado de forma completa, autônoma e segura por qualquer usuário, independentemente de suas capacidades físicas, mentais ou intelectuais~\cite{w3cwai}. Ainda, refere-se à flexibilidade do software em se adaptar às necessidades de cada usuário, suas preferências e suas limitações \cite{camilamaster}. 
Embora a acessibilidade digital esteja relacionada mais especificamente ao atendimento das necessidades específicas de usuários com deficiência, ela impacta a usabilidade global de qualquer software.

Diversos avanços foram alcançados nas últimas décadas nesta área. Um dos avanços que se destaca é a criação de tecnologias assistivas. As tecnologias assistivas ampliam as capacidades dos usuários para que eles possam compreender e operar um produto digital. Por exemplo, um leitor de tela que enuncia para o usuário com deficiência visual todo o conteúdo textual presente na interface do software; ou um sistema que executa as ações dos usuários utilizando apenas interação por voz ao invés do uso de periféricos como \textit{mouse} e teclado, ou até mesmo toques na tela. 

Entretanto, o uso isolado de tecnologias assistivas não são suficientes para tornar um software acessível, pois é preciso que este tenha sido desenvolvido para ser compatível com as diversas tecnologias assistivas existentes. Por exemplo, as aplicações precisam fornecer um conteúdo textual para os elementos não-textuais (como as imagens) para que o leitor de tela possa informar ao usuário qual é a função daquele elemento da interface, seja ela uma possível ação ou uma informação.

Identificar os requisitos necessários para implementar uma interface acessível para os mais diversos tipos de deficiência e as mais diversas tecnologias assistivas é uma tarefa de alto custo. Portanto, diversos padrões e guias de acessibilidade foram criados para direcionar a criação de software e conteúdos digitais acessíveis. Esses padrões apresentam recomendações ou requisitos que precisam ser implementados para que o software seja acessível ao maior número de usuários possíveis. Os detalhes de alguns desses padrões são apresentados na próxima seção. 

%Normalmente a acessibilidade digital é associada exclusivamente às dificuldades motoras, físicas ou sensoriais dos indivíduos com necessidades especiais, sendo desconsiderada importante parcela da população, como por exemplo idosos ou crianças. Uma aplicação que não seja acessível a um indivíduo não pode ser considerada eficaz, eficiente ou agradável ao mesmo \cite{santarosa}.

\section{Padrões de Acessibilidade}

Os padrões de acessibilidade digital visam a fornecer requisitos ou recomendações para a implementação de um software acessível. Alguns deles tem o foco em conteúdos específicos para aplicações Web, outros para aplicações móveis, e outros combinam elementos dessas duas plataformas. Embora alguns padrões sejam específicos para um tipo de plataforma, os princípios apresentados podem ser adaptados e aplicados em diferentes contextos. 
A seguir são apresentados os detalhes de três padrões: o WCAG~\cite{wcag}, o padrão da BBC~\cite{bbc} e o e-MAG~\cite{emag}. 

%Destacam-se como principais promotores dois consórcios internacionais: o W3C (\textit{The World Wide Web Consortium}) e a WAI (\textit{Web Accessibility Initiative}), responsáveis pelo estabelecimento de padrões e protocolos que sistemas computacionais deveriam seguir para serem considerados acessíveis \cite{passerino}.

\subsection{WCAG}

O \textit{Web Content Accessibility Guidelines}, ou WCAG2.1 \footnote{https://www.w3.org/WAI/intro/wcag}, é um documento elaborado pela \textit{Web Accessibility Initiative} (WAI) da \textit{The World Wide Web Consortium} (W3C). 
Este padrão apresenta suas recomendações em formato de declarações testáveis que não se referem a uma tecnologia específica e possui a intenção de tornar o conteúdo \textit{web} mais acessível, com a consciência de que não é capaz de abordar as necessidades de pessoas com todos os tipos, graus e combinações de deficiência.
Inicialmente, o WCAG tinha o foco apenas em conteúdos e em aplicações digitais, mas em sua nova versão foram incorporados elementos relacionados à plataforma móvel.

As recomendações do WCAG são classificadas em três níveis de conformidade: A (o mais baixo), AA e AAA (o mais elevado). Além disso, as recomendações estão divididas em quatro princípios fundamentais: \emph{perceptível}, pois a informação deve ser apresentada na interface de forma a garantir que qualquer usuário possa percebê-la;
\emph{operável}, pois todos os componentes de navegação da aplicação devem ser operáveis por qualquer usuário;
\emph{compreensível}, pois todo usuário deve conseguir entender a informação passada pela interface;
\emph{robusto}, pois a interface deve ser robusta o suficiente para que possa ser interpretada por qualquer usuário, utilizando ou não tecnologias assistivas.
As Tabelas~\ref{tab:perceptivel}, ~\ref{tab:operavel} e ~\ref{tab:compreensivelrobusto}  mostram as recomendações para cada princípio do WCAG.

\begin{table}[]
\center
\caption{Diretrizes e requisitos do princípio Perceptível}
\label{tab:perceptivel}
\begin{tabular}{lll}
\textbf{ID} & \textbf{DIRETRIZ}      & \textbf{REQUISITO}   \\
\hline
 1.1.1           & Alternativas de texto  & Conteúdo Não Textual                                                                 \\
 1.2.1           & Mídia baseada no tempo & Apenas Áudio e apenas Vídeo (Pré-gravado)                                            \\
 
 1.2.2           & Mídia baseada no tempo & Legendas (Pré-gravado)                                                               \\
 
 1.2.3           & Mídia baseada no tempo & Audiodescrição ou Mídia Alternativa (Pré-gravado)                                    \\
 
 1.2.4           & Mídia baseada no tempo & Legendas (Ao Vivo)                                                                  \\
 
 1.2.5           & Mídia baseada no tempo & Audiodescrição (Pré-gravado)                                                        \\
 
 1.2.6           & Mídia baseada no tempo & Linguagem de sinais (pré-gravada)                                                   \\
 
 1.2.7           & Mídia baseada no tempo & Audiodescrição estendida (pré-gravado)                                              \\
 
 1.2.8           & Mídia baseada no tempo & Mídia Alternativa (Pré-gravado)                                                     \\
 
 1.2.9           & Mídia baseada no tempo & Apenas áudio (Ao Vivo)                    \\
 
 1.3.1           & Adaptável              & Informações e Relações                                                               \\
 
 1.3.2           & Adaptável              & Sequência com Significado                                                            \\
 
 1.3.3           & Adaptável              & Características Sensoriais                                                           \\
 
 1.3.4           & Adaptável              & Orientação                                                                          \\
 
 1.3.5           & Adaptável              & Identificar o objetivo de entrada                                                   \\
 
 1.3.6           & Adaptável              & Identificar o propósito                                                             \\
 
 1.4.1           & Distinguível           & Utilização de Cores                                                                  \\
 
 1.4.2           & Distinguível           & Controle de Áudio                                                                    \\
 
 1.4.3           & Distinguível           & Contraste (Mínimo)                                                                  \\
 
 1.4.4           & Distinguível           & Redimensionar texto                                                                 \\
 
 1.4.5           & Distinguível           & Imagens de Texto                                                                    \\
 
 1.4.6           & Distinguível           & Contraste (Melhorado)                                                               \\
 
 1.4.7           & Distinguível           & Som baixo ou sem som de fundo                                                       \\
 
 1.4.8           & Distinguível           & Apresentação Visual                                                                 \\
 
 1.4.9           & Distinguível           & Imagens de texto (sem exceção)                                                      \\
 
 1.4.10          & Distinguível           & Reflow                                                                              \\
 
 1.4.11          & Distinguível           & Contraste sem texto                                                                 \\
 
 1.4.12          & Distinguível           & Espaçamento de texto                                                                \\
 
 1.4.13          & Distinguível           & Conteúdo sobre do foco do mouse ou do teclado                                       \\
\hline
\end{tabular}
\end{table}


\begin{table}[]
\center
\caption{Diretrizes e requisitos do princípio Operável}
\label{tab:operavel}
\begin{tabular}{lll}
\textbf{ID} & \textbf{DIRETRIZ}      & \textbf{REQUISITO}                              \\
\hline
 2.1.1           & Operável               & Teclado                                                                              \\
 
 2.1.2           & Operável               & Sem Bloqueio do Teclado                                                              \\
 
 2.1.3           & Operável               & Teclado (sem exceção)                                                               \\
 
 2.1.4           & Operável               & Atalhos para teclas de caracteres                                                    \\
 
 2.2.1           & Tempo suficiente       & Ajustável por limite de tempo                                                        \\
 
 2.2.2           & Tempo suficiente       & Colocar em Pausa, Parar, Ocultar                                                     \\
 
 2.2.3           & Tempo suficiente       & Sem limite de tempo                                                                 \\
 
 2.2.4           & Tempo suficiente       & Interrupções                                                                        \\
 
 2.2.5           & Tempo suficiente       & Re-autenticação (nova autenticação)                                                 \\
 
 2.2.6           & Tempo suficiente       & Tempos Limite                                                                       \\
 
 2.3.1           & Convulsões             & Três Flashes ou Abaixo do Limite                                                     \\
 
 2.3.2           & Convulsões             & Três flashes                                                                        \\
 
 2.3.3           & Convulsões             & Animação de Interações                                                              \\
 
 2.4.1           & Navegável              & Ignorar Blocos                                                                       \\
 
 2.4.2           & Navegável              & Página com Título                                                                    \\
 
 2.4.3           & Navegável              & Ordem do Foco                                                                        \\
 
 2.4.4           & Navegável              & Finalidade do Link (Em Contexto)                                                     \\
 
 2.4.5           & Navegável              & Várias Formas (múltiplos caminhos)                                                  \\
 
 2.4.6           & Navegável              & Cabeçalhos e Rótulos                                                                \\
 
 2.4.7           & Navegável              & Foco Visível                                                                        \\
 
 2.4.8           & Navegável              & Localização                                                                         \\
 
 2.4.9           & Navegável              & Objetivo do link (apenas link)                                                      \\
 
 2.4.10          & Navegável              & Cabeçalhos das Seções                                                               \\
 
 2.5.1           & Modalidades de Entrada & Gestos de Ponteiro                                                                   \\
 
 2.5.2           & Modalidades de Entrada & Cancelamento de ponteiro                                                             \\
 
 2.5.3           & Modalidades de Entrada & Rótulo em nome                                                                       \\
 
 2.5.4           & Modalidades de Entrada & Atuação do movimento                                                                 \\
 
 2.5.5           & Modalidades de Entrada & Tamanho do Alvo                                                                     \\

 2.5.6           & Modalidades de Entrada & Mecanismos de Entrada Simultâneos                                                   \\
 \hline
 \end{tabular}
\end{table}



\begin{table}[]
\center
\caption{Diretrizes e requisitos do princípio Compreensível}
\label{tab:compreensivelrobusto}
\begin{tabular}{lll}
\textbf{ID-DIR} & \textbf{DIRETRIZ}      & \textbf{REQUISITO}                                             \\
\hline
 3.1.1           & Legível                & Idioma da Página                                                                     \\
 
 3.1.2           & Legível                & Idioma das Partes                                                                   \\
 
 3.1.3           & Legível                & Palavras incomuns                                                                   \\
 
 3.1.4           & Legível                & Abreviações                                                                         \\
 
 3.1.5           & Legível                & Nível de Leitura                                                                    \\
 
 3.1.6           & Legível                & Pronúncia                                                                           \\
 
 3.2.1           & Previsível             & Em Foco                                                                              \\
 
 3.2.2           & Previsível             & Em Entrada                                                                           \\
 
 3.2.3           & Previsível             & Navegação Consistente                                                               \\
 
 3.2.4           & Previsível             & Identificação Consistente                                                           \\
 
 3.2.5           & Previsível             & Alteração no pedido                                                                 \\
 
 3.3.1           & Assistência de entrada & Identificação do Erro                                                                \\
 
 3.3.2           & Assistência de entrada & Rótulos ou Instruções                                                                \\
 
 3.3.3           & Assistência de entrada & Sugestão de Erro                                                                    \\
 
 3.3.4           & Assistência de entrada & Prevenção de Erros (Legal, Financeiro, Dados)                                       \\
 
 3.3.5           & Assistência de entrada & Ajuda disponível                                                                    \\
 
 3.3.6           & Assistência de entrada & Prevenção de Erros (Todos)                                                          \\
 \hline
%\textbf{ID} & \textbf{DIRETRIZ}      & \textbf{REQUISITO} \\
\hline
4.1.1           & Compatível             & Análise (código)                                                                     \\
 
4.1.2           & Compatível             & Nome, Função, Valor                                                                  \\
 
4.1.3           & Compatível             & Mensagens de Status           \\                                                     
\hline
\end{tabular}
\end{table}



\subsection{\textit{BBC Accessibility Guidelines}}

A BBC - (\textit{British Broadcasting Corporation})~\cite{bbc} é uma empresa pública líder mundial em serviços de transmissão, incluindo canais de televisão e rádio, bem como transmissões via internet. 
Com a intenção de manter seus conteúdos disponíveis para o maior número de pessoas, foram definidas diretrizes que devem ser seguidas para os conteúdos disponibilizados pela empresa, e que se tornaram uma das referências globais. O conjunto de práticas recomendadas podem ser aplicadas a conteúdos Web móvel, aplicações híbridas e nativas. As diretrizes do padrão da BBC estão divividas em 11 categorias. A seguir é apresentada uma visão geral de cada uma das diretrizes deste padrão e organizadas de acordo com cada categoria:


\begin{itemize}
	\item Áudio e Vídeo:
		\subitem - Alternativas para conteúdos de vídeo e áudio: quando possível, devem ser disponibilizados conteúdos alternativos, como por exemplo em legendas, linguagem de sinais e transcrições, incorporados à midia.
		\subitem - Autoplay: elementos de áudio não devem iniciar automaticamente, exceto se o usuário tiver conhecimento prévio ou que sejam fornecidos botões de controle (pausa, parada e mudo).
		\subitem - Metadados: Os metadados relevantes devem ser disponibilizados para todas as mídias.
		\subitem - Controle de Volume: em havendo música de fundo, sons de ambiente ou efeitos sonoros narrativos e de edição, deverão ser disponibilizados controles de volume separadamente.
		\subitem - Conflito de áudio: não devem ocorrer conflitos dos áudios da tecnologia de assistência nativa com os sons narrativos em jogos ou mídias interativas.
	\item \textit{Design}:
		\subitem - Contraste de cores: deve haver um contraste mínimo entre a cor do texto e a cor do conteúdo do plano de fundo.
		\subitem - Cor e significado: informação ou significado não deve ser transmitido apenas através da diferença de cores.
		\subitem - Estilo e leitura: o conteúdo principal deve ser acessível, mesmo quando o estilo não for suportado ou for removido intencionalmente pelo usuário.
		\subitem - Tamanho de itens clicáveis: devem ser grandes o suficiente para que o usuário possa tocar com precisão.
		\subitem - Espaçamento: deve haver um espaço inativo mínimo entre os itens clicáveis.
		\subitem - Conteúdo dimensionável: o usuário deve poder controlar o dimensionamento da fonte e a escala (\textit{zoom}) da tela de interface do usuário.
		\subitem - Elementos acionáveis: deve ser possível a distinção clara de \textit{links} e outros elementos acionáveis.
		\subitem - Foco: todos os elementos devem continuar visíveis, independentemente do foco utilizado pelo usuário.
		\subitem - Consistência: a experiência do usuário deve ser consistente, manter a mesma lógica e linguagem, o que facilitará ao usuário prever a próxima etapa.
		\subitem - Escolha: as interfaces do usuário devem prover múltiplas formas de interação com o seu conteúdo.
		\subitem - Ajustabilidade: mídias interativas, incluindo jogos, devem ser ajustáveis pelo usuário de acordo com sua capacidade (habilidade) e preferência.
		\subitem - Cintilação: não deve haver conteúdo que pisque mais do que 3 vezes em um período de 1 segundo. Este tipo de cintilação pode causar fadiga ocular, tonturas, dores de cabeça, enxaqueca, náusea, podendo chegar a vertigem e até mesmo convulsões em casos específicos.
	\item Editorial:
		\subitem - Títulos consistentes: devem ser utilizados em \textit{web sites} e aplicações nativas, de forma a facilitar o entendimento do conteúdo completo, tornando-o familiar, principalmente para usuários que se utilizam de leitores de tela.
		\subitem - Indicação do idioma: deve estar especificado o idioma de uma página ou aplicação, sendo que alterações no conteúdo devem ser indicadas ao usuário.
		\subitem - Instruções: quando necessário, instruções adicionais devem ser fornecidas de forma a auxiliar o conteúdo disponibilizado em formato de áudio e vídeo.
	\item Foco:
		\subitem - Elementos focáveis: todos os elementos interativos devem ser focáveis. Elementos inativos não devem ser focáveis.
		\subitem - Armadilhas de teclado: não deve existir armadilhas de teclado, de forma que o usuário possa controlar a interface através de um teclado ou mesmo uma entrada que não possua um "ponteiro".
		\subitem - Ordem do conteúdo: a ordem do conteúdo deve ser lógica, facilitando o entendimento do mesmo principalmente por usuários que se utilizam de tecnologia assistiva.
		\subitem - Ordem do foco: o conteúdo clicável deve ser navegável em uma sequência entendível. Por exemplo: navegar em um formulário sem ordem lógica do foco tornará o mesmo desorientador para um usuário com leitor de tela.
		\subitem - Interações do usuário: Ações devem desencadear outra interação apropriada, de acordo com o método de entrada de dados pelo usuário, como por exemplo mouse, teclado ou mesmo outros controladores.
		\subitem - Métodos de entrada alternativos: Devem ser suportados métodos de entrada alternativos, como por exemplo telas em braille ou simplesmente um teclado.
	\item Formulários:
		\subitem - Rótulos dos controles dos formulários: todos os controles dos formulários devem possuir rótulos exclusivos e disponíveis para tecnologias assistidas, facilitando o entendimento.
		\subitem - Entrada de dados: deve ser claramente indicado e suportado um formato de entrada de dados padrão, facilitando o usuário de entender e acertar a entrada na primeira vez.
		\subitem - \textit{Layout} dos formulários: os rótulos devem ser colocados próximos dos controles do formulário, reduzindo o risco do usuário de se desorientar.
		\subitem - Agrupamento de elementos: os controles, rótulos e outros elementos do formulário devem estar adequadamente agrupados, o que reduzirá o número de passos e complexidade de preenchimento principalmente por usuários que se utilizam de tecnologia assistiva.
		\subitem - Foco manuseável: O foco ou o contexto não devem mudar automaticamente durante a entrada de dados, mas sim apenas com uma ação do próprio usuário.
	\item Imagens:
		\subitem - Imagens de texto: devem ser evitadas, já que se trata de uma forma inflexível de passagem de informação, estando indisponível para tecnologias assistidas.
		\subitem - Imagens de fundo: as imagens de fundo que contenham informações devem ser evitadas ou conter uma alternativa acessível adicional, já que não estão disponíveis em tecnologias assistidas.
	\item \textit{Links}:
		\subitem - \textit{Links} descritivos: O texto do \textit{link} ou do item de navegação deve descrever exclusivamente seu destino ou função.
		\subitem - \textit{Links} para formatos alternativos: \textit{Links} para formatos alternativos devem indicar
		que uma página alternativa será aberta, caso contrário desorientará usuários com dificuldades cognitivas ou que se utilizam de tecnologia assistida.
		\subitem - Combinação de \textit{links} repetidos: \textit{Links} repetidos para o mesmo recurso devem ser combinados em um único \textit{link}, o que auxiliará os usuários a navegar rapidamente pelo conteúdo, especialmente aqueles que dependem de tecnologia assistida.
	\item Notificações:
		\subitem - Notificações inclusivas: devem ser visíveis e audíveis.
		\subitem - Notificações do sistema operacional: devem ser utilizadas as notificações padrão do sistema operacional quando disponíveis e de forma apropriada.
		\subitem - Mensagens de erro e correção: devem ser claras.
		\subitem - \textit{Feedback} e assistência: \textit{Feedback} ou assistência não críticos devem ser fornecidos quando apropriado.
	\item \textit{Scripts} e Conteúdos Dinâmicos:
		\subitem - Funcionamento progressivo: Aplicações e sites devem ser criados de forma a funcionar de maneira progressiva, garantindo uma experiência funcional para todos os usuários.
		\subitem - Controle de mídia: em apresentações de mídias devem existir botões para controles de pausa, parada ou mesmo ocultação dos controles.
		\subitem - Atualização de página: As atualizações automáticas de página não devem ser usadas sem prévio aviso, podendo impactar tecnologias assistidas, como por exemplo leitores de tela.
		\subitem - Tempos de espera: os tempos de resposta devem ser ajustáveis, algumas pessoas podem não ser capazes de responder dentro do tempo esperado.
		\subitem - Controle de entrada: Interações de entrada devem ser adaptáveis, de forma a permitir que usuários com deficiências motoras possam ajustar.
	\item Estrutura:
		\subitem - Título único	para páginas ou telas: Todas as páginas ou telas devem conter um único e identificável título.
		\subitem - Cabeçalho: O conteúdo deve fornecer uma estrutura lógica e hierárquica de cabeçalho, de acordo com o que é suportado pela plataforma.
		\subitem - Contêiners e marcadores: contêineres devem ser usados para descrever a estrutura da página ou tela, de acordo com o que é suportado pela plataforma.
		\subitem - Grupos de elementos: controles, objetos e elementos de interface agrupados devem ser representados como um único componente acessível.
	\item Textos Equivalentes:
		\subitem - Alternativas para conteúdos não textuais: deve haver uma breve descrição da intenção ou propósito do conteúdo, imagem, objeto ou elemento.
		\subitem - Conteúdo decorativo: imagens decorativas devem ser escondidas de tecnologias assistidas.
		\subitem - Dicas e informações complementares: as dicas de ferramentas não devem repetir o texto do link ou outras alternativas.
		\subitem - Tarefas, marcas e propriedades: elementos devem conter propriedades de acessibilidade apropriados.
		\subitem - Formatação visual: não deve ser utilizado apenas a formatação visual para transmitir um determinado significado ou mensagem.
\end{itemize}

Conforme abordado em \cite{camilamaster}, apesar das diretrizes de acessibilidade propostas pela BBC serem compreensíveis e de fácil interpretação, pode-se dizer que a W3C possui diretrizes não cobertas pela BBC, conforme apresentado abaixo:
\begin{itemize}
	\item Montante de informações: em telas de aplicações móveis, é fundamental a redução da quantidade de informações apresentadas na tela se comparadas com as versões de \textit{desktop}.
	\item Posição dos títulos: os formulários devem possuir seus títulos dispostos acima, ao invés de estarem ao lado da página.
	\item Teclado: todas as funcionalidades devem ser operáveis sem a necessidade da utilização de um teclado.
	\item Gestos: o seu emprego deve ser fácil, sem a necessidade de percorrer caminhos específicos.
	\item Posição dos elementos interativos: devem estar posicionados de forma que o usuário possa identificá-los facilmente.
	\item Orientação da tela: aplicações móveis devem suportar as duas orientações da tela, sendo possível de ser percebida facilmente por tecnologias assistidas.
	\item Posicionamento dos elementos: as informações importantes devem estar visíveis, sem que haja a necessidade de rolagem da tela para identificá-las.
	\item Instruções: é fundamental a inserção de títulos ou instruções que auxiliem o usuário a inserir as informações requeridas.
	\item Ajuda: devem estar disponíveis e de fácil identificação, mesmo com tecnologias assistidas.
	\item Facilidade na entrada de dados: pode haver a substituição de volumes de texto de entrada por menus de seleção, botões \textit{radio}, caixas de seleção ou mesmo por entrada automática de dados, desde que estes estejam claramente informados ao usuário e contemplados por tecnologias assistidas.
\end{itemize}

Apesar do alto número de padrões e diretrizes definidos tanto pela BBC quanto pela W3C, é de conhecimento que estas definições não são completas, e que outros grupos ou companhias ou consórcios podem definir novos padrões de acordo com as necessidades de seus usuários. 
%Segundo \cite{meloihc2004}, a avaliação da usabilidade de um software é definida pela verificação de sua acessibilidade relacionada ao seu contexto de uso, às atividades que o mesmo apoia, necessidades e preferências dos usuários envolvidos.  



\subsection{eMAG - Modelo de Acessibilidade em Governo Eletrônico}

Em 2004, o governo brasileiro iniciou os trabalhos relativos às definições dos padrões de acessibilidade e criou o Modelo de Acessibilidade Brasileiro, o eMAG~\cite{emag}, com base em padrões internacionais como o WCAG. 
Este modelo foi proposto com o objetivo de ser um norteador para o desenvolvimento de portais Web acessíveis como forma de promover a inclusão social na população brasileira. 
Na data de 07 de maio de 2007, a Portaria nº 3 institucionalizou o padrão eMAG, tornando-o obrigatório no âmbito do governo federal ou de instituições públicas ou privadas que possuem relação com organizações federais. Posteriormente, em 2015, foi sancionada a lei Brasileira de nº 13.146, que estabelece as normas gerais de acessibilidade, incluindo as áreas de sistemas de informação e conteúdos digitais.

A versão 3.1 do eMAG, de Abril de 2014, define 45 recomendações de acessibilidade que estão divididas em seis seções. Diferentemente da WCAG, o eMAG não define prioridades ou níveis de conformidade para suas recomendações. A seguir é apresentada uma visão geral das recomendações deste padrão brasileiro:
\begin{itemize}
	\item[1] Marcação: as 9 recomendações desta seção visam orientar os desenvolvedores a organizar as camadas do código, respeitando os padrões \textit{web} de forma lógica e semântica. Além disso direciona a disponibilização de \textit{links} diretos que facilitarão principalmente a navegação pelo usuário que se utiliza de tecnologias assistidas, bem como a não abertura de instâncias (abas ou janelas) sem a prévia solicitação do usuário.
	
	\item[2] Comportamento (DOM): as 7 recomendações descritas na seção visam orientar a utilização e controle da navegação. Independentemente do tipo de entrada utilizado (teclado, \textit{mouse} ou mesmo \textit{touchscreen}), os objetos programáveis devem ser acessíveis e as páginas não devem ser atualizadas ou redirecionadas automaticamente sem a autonomia do usuário. Nesta seção também se aborda o tema cintilação, que podem causar sérios danos a usuários sensíveis.
	
	\item[3] Conteúdo/Informação: as 12 recomendações desta seção visam garantir que o usuário possuirá as informações relevantes de cada elemento da tela, como por exemplo idioma da página ou conteúdo, textos descritivos e claros de objetos, imagens e links. Palavras incomuns, abreviaturas e siglas também devem conter suas descrições claramente. Por fim, direciona que estas informações estejam disponíveis em tecnologias assistidas.

	\item[4] Apresentação/Design: nesta seção estão as 4 recomendações relacionadas à disponibilização visual dos objetos da aplicação. São citados taxa de contraste mínimo, não utilização apenas da cor como elemento para transmissão de informações, redimensionar a tela sem perdas de funcionalidades e por fim possibilitar que o elemento com foco esteja destacado dos demais.
	
	\item[5] Multimídia: contemplando 5 recomendações, esta seção tem a função de direcionar que hajam alternativas e controles para o usuário na apresentação de informações de conteúdos multimídia. Legendas, audiodescrição, áudios alternativos, botões de controle são citados como fundamentais para o atendimento a esta categoria.	
	
	\item[6] Formulário: as 8 recomendações visam disponibilizar formulários acessíveis, com textos descritivos, etiquetas e/ou orientações sobre cada item, manutenção da leitura lógica através de tecnologia assistida, evitando alterações automáticas no contexto. Caso hajam erros nas inserções de dados, deverá haver textos explicativos e que sinalizem os locais a serem corrigidos. Por fim, estratégias de segurança também são abordadas nesta seção.
	
\end{itemize}

No detalhamento das recomendações mais complexas, são apresentados exemplos que auxiliam no entendimento e direcionamento dos desenvolvedores. Em decorrência do eMAG ter sido concebido baseado no WCAG, para cada recomendação existe o respectivo relacionamento com a WCAG.


 
%\section{Novos Desafios}
%Com o avanço da tecnologia móvel, um novo panorama foi criado para a interface humano computador. As páginas Web acessadas em \textit{desktops} (computadores pessoais e fixos nas residências ou locais de trabalho) passaram a ser acessadas através de celulares, com telas menores e com localização móvel. As tarefas envolvendo o computador há 10 anos que eram realizadas em locais muitas vezes silenciosos, passou a fazer parte do cotidiano, sendo acessado em metrôs, ônibus e até mesmo nas ruas. A utilização do software pelos usuários, anteriormente disponibilizado para instalação através de midias físicas vendidas em lojas, passou a ser disponibilizado praticamente de forma instantânea nas lojas de aplicativos (ex: \textit{Google Play}), alterando a forma e periodicidade em que os usuários instalam ou atualizam seus \textit{apps} \cite{nayebi}, impulsionando o desenvolvimento em formato Ágil.

%Esta mudança de características trouxe novos desafios ao cotidiano dos desenvolvedores, funcionalidades envolvendo geolocalização, as resoluções das telas e respectivos objetos precisam ser reconsiderados, a luminosidade no ambiente do usuário passou a ser uma importante variável.

\section{Trabalhos Relacionados}

O objetivo desta proposta de pesquisa é investigar se as avaliações dos aplicativos publicadas nas lojas oficiais estão relacionadas à acessibilidade da aplicação. Com exceção dos estudos iniciais realizados no início deste projeto~\cite{ihc2019} e de uma pesquisa utilizando os dados disponibilizados por ele~\cite{rochestertamjeed} com a intenção de automatizar a identificação automática de avaliações relacionadas a acessibilidade, nenhum outro estudo foi encontrado com este mesmo objetivo. 
Portanto, nesta seção são mencionados os estudos publicados que fazem análise de outros aspectos relacionados às avaliações dos usuários publicadas nas lojas de aplicativos. 

%evolução de apps (evoluem rapidamente), geralmente é ágil, novas versões surgem a cada semana, citar referência, citar estudos do panichella, mario linhares, e assim por diante
%No contexto relacionado aos estudos já realizados sobre avaliações em lojas de aplicativos, podemos elencar vários artigos, porém os únicos centrados em discussões envolvendo acessibilidade tratam-se do artigo referente ao estudo piloto desta pesquisa \cite{ihc2019} e do estudo evolutivo deste artigo abordando uma análise automatizada \cite{rochestertamjeed}. Não identificamos estudos correlacionando avaliações, solicitações de modificações e alterações envolvendo acessibilidade.

Com o advento das lojas de aplicativos (e.g. \textit{Google Play Store} e \textit{Apple Store}),
a relação entre desenvolvedor e usuário começou a ser alterada. Avaliações de usuários passaram a ser postadas publicamente causando impactos tanto nas notas da aplicação (\textit{ratings}) quanto no número de vezes que o software é baixado (\textit{downloads}), e desta forma tornando-se para o desenvolvedor uma importante fonte de compreensão do seu público \cite{Pagano2013userfeedback}. De acordo com \cite{Fu2013whypeoplehate}, baseado em mais de 13 milhões de avaliações, foi possível uma análise da exigência dos usuários de acordo com a categoria da aplicação, observando-se que os mesmos tendem a ser mais tolerantes para a categoria jogos do que para demais aplicações.

A oportunidade do cliente de expôr sua opinião com textos livres deu origem a um grande número de demandas~\cite{Mcilroy2016analyzing}, o que traz aos desenvolvedores a dificuldade de identificar as necessidades a serem tratadas nas próximas entregas, especialmente para softwares populares com alta quantidade de opiniões postadas. Este cenário tem promovido pesquisas sobre formas de sumarização \cite{Iacob2013retrieving,Iacob2014online,Fu2013whypeoplehate} e interpretação destes textos, incluindo o emprego conjunto de diferentes técnicas para aprendizado de máquina \cite{Panichella2015how}.

Em \cite{Palomba2015userreviews}, \cite{Palomba2018crowdsourcing} e \cite{Li2018MobileAE} foram realizadas pesquisas que associaram as avaliações às disponibilizações de versões do software (\textit{releases}). As conclusões foram de que as notas aumentam quando há implementações que visam atender às opiniões dos usuários. Mesmo o pequeno volume de textos informativos trata-se de uma valiosa fonte de informações, que permite tanto a correção de erros de difícil identificação nos testes quanto a implantação de novas funcionalidades e recursos não-funcionais.

Embora diversos trabalhos tenham sido realizados para entender os tipos de demandas dos usuários e de que forma elas eram tratadas e utilizadas para o planejamento das novas versões dos aplicativos~\cite{Iacob2013retrieving,Pagano2013userfeedback,Iacob2014online,Mcilroy2016analyzing,Sorbo2017surf,Ciurumelea2017analyzing,Li2018MobileAE,Pelloni2018becloma,Panichella2015how}, nenhum estudo fez uma diferenciação ou um detalhamento sobre os aspectos de acessibilidade abordados pelos usuários. 


%APRESENTAÇÃO DOS DETALHES DOS TRABALHOS RELACIONADOS A ESTA PESQUISA
%INCLUSÃO EM 15/11, ANTES DA ENTREGA FINAL DA QUALIFICAÇÃO

A seguir são apresentados detalhes dos estudos relacionados a esta pesquisa.

\subsection{\textit{User Feedback in the AppStore: An Empirical Study}}
O estudo \cite{Pagano2013userfeedback} de 2013 já apresentava a importância das avaliações dos usuários na engenharia de requisitos. O estudo exploratório considerou mais de um milhão de avaliações, identificando que a maior parte era enviada logo após o lançamento da aplicação, com redução rápida ao longo do tempo. Os conteúdos dos textos causam impacto direto no número de \textit{downloads}, sendo que as avaliações negativas referentes às deficiências da aplicação são tipicamente destrutivas e perdem detalhes de contexto e experiência do usuário.

Trata-se de um artigo de 2013, quando ainda estava sendo desvendado o relacionamento entre desenvolvedores, lojas de aplicativos e usuários que podem expôr suas avaliações abertamente, ou seja, formadores de opinião.

O artigo cita a observação de que o tamanho dos textos é maior para aplicações mais caras. Além disso, claramente os usuários tornam-se raivosos (enviando textos com insultos), principalmente quando houve compra da aplicação.

Em 2013, quando da elaboração do estudo, conclui-se que as avaliações são uma ótima fonte para os desenvolvedores entenderem seu público, porém foi salientada a falta de ferramenta que permita a análise e agrupamento das avaliações que facilitem a mineração das informações pelos desenvolvedores.


\subsection{\textit{How can i improve my app? Classifying user reviews for software maintenance and evolution}}
Em \cite{Panichella2015how} é apresentado um estudo sobre as avaliações de aplicações móveis utilizando a combinação de três técnicas: Processamento de Linguagem Natural (PLN), Análise de Texto (AT) e Análise de Sentimento (AS). Os resultados sugerem ser possível a utilização destas informações para a evolução das aplicações pelos desenvolvedores.

As principais conclusões referentes às técnicas são que a combinação das mesmas produz melhor resultado na classificação e entendimento dos textos, e que a Análise de Sentimento sozinha não apresenta valores significativos, porém agregando valor quando combinada com as demais.

A metodologia do estudo consistiu em: (i) definir uma taxonomia para identificar os textos relacionados à manutenção e evolução do software; (ii) extração dos desejos dos usuários, considerando tanto novas funcionalidades quanto correções; (iii) aprendizado das técnicas e execução das mesmas combinadas; (iv) avaliação dos resultados.
A taxonomia foi gerada manualmente por 2 autores do trabalho, e revisadas por um aluno de PhD, a partir de 300 e-mails produzidos em lista de discussão sobre 2 projetos de código fonte aberto, bem como se utilizando de técnica qualitativa publicada em 1967. Para a avaliação foi utilizada base de dados de trabalho anterior contendo as principais aplicações disponíveis nas lojas de aplicativos. Para a avaliação da metodologia foi feita a classificação manual por dois autores utilizando uma amostragem de 18\% (1.421) das avaliações para um determinado software, que possuía 7.696 avaliações no total. Apenas 31 casos (2,81\%) foram categorizados como "outros".

Os resultados sugerem que os usuários utilizam padrões para respostas quando a intenção é reportar um problema, porém quando a intenção é sugerir uma nova funcionalidade são identificados diferentes padrões, o que dificulta a identificação. Além disso AS sozinha não apresenta valores significantes, no entanto agrega valor quando em combinação com outras técnicas.

Como ameaça de validade da base de estudo, pode-se citar a má interpretação do julgamento manual das classificações, porém mitigado com processo realizado em 2 etapas, por pessoas distintas, resultando em questionamentos para apenas 2,81\% dos casos.

Como ameaças à validade interna citou-se inicialmente que os textos podem ser classificados em mais de uma intenção. Para isso eles analisaram o histórico das comunicações bem como ordenaram as classificações, permitindo alocar na de pior cenário. Outra ameaça à validade interna é o sobreajuste, cujo modelo se adequa muito bem ao oráculo, porém não necessariamente apresentará os mesmos resultados para o conjunto completo. Para reduzir esta ameaça foram executados os processos de aprendizagem sobre 20\% das amostras, repetindo o processo 100 vezes e aplicando validação cruzada 10 vezes.

Como ameaças externas citou-se que o processo foi executado para um grupo específico de aplicações. Para reduzir esta ameaça foram selecionados casos de diferentes categorias, de duas lojas distintas.

\subsection{\textit{Analyzing reviews and code of mobile apps for better release planning}}
No artigo \cite{Ciurumelea2017analyzing} é apresentado o método \textit{User Request Referencer} (URR) que permite ao desenvolvedor a obtenção mais rápida do código de seu software relacionado à avaliação do usuário. O estudo validou a possibilidade de categorizar a avaliação tanto em alto quanto baixo nível, salientando que o processo ainda necessitava de mais treinamentos.

Foi estimada uma economia de tempo de até 75\% para estas identificações do código, facilitando o processo para os casos em que as avaliações estão melhor explicadas.

O método URR se utiliza dos seguintes passos: (i) Definição da Taxonomia; (ii) Classificação das avaliações dos usuários; (iii) Localização do código fonte.

A definição da taxonomia foi feita a partir de amostragens de 39 aplicações. Aqueles que possuíam muitas avaliações a quantidade foi limitada a 200. O volume total de avaliações analisadas manualmente foi de 1.566.

Para a classificação de alto nível, tanto as avaliações positivas quanto as de classificações negativas são importantes para o desenvolvedor, o qual necessitará atuar mais fortemente ou poderá decidir atuar em outra funcionalidade.
Para a classificação de baixo nível, a mesma é importante para o desenvolvedor identificar com precisão o ponto merecedor de atenção pelo desenvolvedor.

A extração das funcionalidades foi feita utilizando Aprendizado de Máquina, de acordo com a taxonomia definida anteriormente. Foi decidido não considerar a avaliação em estrelas dada pelo usuário. Considerou-se a razão de 20\% das avaliações como sendo base para aprendizagem.

Para a localização do código fonte, foi utilizado \textit{Apache Lucene API}, de acordo com as etapas seguintes: (i) obtenção do código fonte de cada software; (ii) pré processamento de cada código fonte e avaliações considerando \textit{stop word} em inglês e reduzindo distúrbios; (iii) indexação dos conteúdos do código fonte e avaliações; (iv) pré localização, feita através da criação de uma base de dados utilizando classificações das avaliações e localização das mesmas dentro do padrão de código fonte; (v) realização de pesquisa por relevência, onde neste teste foram obtidos resultados de 30\%, com previsão de melhoras em trabalhos futuros.

As perguntas do artigo foram:

RQ1: Até que ponto o Referenciador de Solicitações do Usuário organiza as avaliações de acordo com tarefas significativas de manutenção e evolução para desenvolvedores?

RQ2: O URR recomenda corretamente os artefatos de software que precisam ser modificados para lidar com solicitações de usuários e reclamações?
Base utilizada de 7.754 avaliações envolvendo 39 aplicações de código fonte abertos, que resultaram em 7.242 classes e 940.051 linhas de código Java.

Foram realizados 2 experimentos:
Experimento I foi feito para responder à RQ1. Executou-se o URR com 20\% com textos e foi então solicitada a classificação manual dos mesmos, informando se o URR acertou ou não. Desta forma não foi utilizado o mesmo conjunto de 1.566 avaliações utilizadas na elaboração da taxonomia.
Experimento II foi feito para responder à RQ2. Foi então feito um oráculo com 91 avaliações.

As conclusões para a RQ1 é que é possível categorizar tanto em alto quanto baixo nível, no entanto há a necessidade de melhorar o treinamento para alguns casos. Foi estimado uma possível economia de tempo de até 75\%, sendo as categorias mais importantes Uso, Recursos e Compatibilidade.
Para RQ2 conclui-se que URR alcança resultados promissores, sendo que as melhores avaliações são mais fáceis de vincular.

\subsection{\textit{SURF: Summarizer of User Reviews Feedback}}
Neste estudo \cite{Sorbo2017surf} foi apresentada uma ferramenta denominada SURF (\textit{Summarizer of User Reviews Feedback}) que se propõe analisar, classificar e sumarizar os textos das avaliações dos usuários com a intenção de facilitar a visualização pelos desenvolvedores. No estudo empírico participaram 12 desenvolvedores da Suíça, Itália e Holanda, que avaliaram a usabilidade da ferramenta. Um total de 12 aplicações e 2.622 textos de avaliações foram utilizados, de diferentes lojas e categorias de aplicativos.

A primeira etapa do processo foi a classificação da intenção dos usuários, utilizando Processamento em Linguagem Natural (PLN) e Análise de Sentimento.
A segunda etapa classifica em diferentes agrupamentos, também se utilizando de através de PLN.
A terceira etapa define então uma pontuação para cada avaliação, considerando vários itens como probabilidade da palavra estar em outras avaliações, número de palavras da avaliação e importância destas para a classificação. Apenas as avaliações com maiores pontuações são consideradas para a etapa seguinte, sendo que o corte foi de 70\%.
Por fim a etapa de geração da sumarização. A ferramenta gera um arquivo do tipo \textit{XML} contendo as informações das avaliações, classificações e agrupamentos.

O artigo explica como utilizar a ferramenta, informa que foi disponibilizada em versão com interface para usuário de forma a facilitar tanto a inserção quanto a apresentação dos dados. 

A avaliação foi realizada com 12 aplicações (subconjunto de um total de 17) e 2.622 avaliações, de diferentes lojas e categorias de aplicativos.
Os resultados foram disponibilizados para pesquisadores, profissionais da indústria de software e desenvolvedores aleatoriamente, sendo que 9 participantes (de um total de 12) julgaram que os resultados auxiliam no entendimento dos textos. Foi estimada uma economia de tempo de 50\% e assertividade de 92\%.

As conclusões finais foram que futuramente a ferramenta poderia ser expandida para identificar a parte do código e evolução da extração dos dados.

\subsection{\textit{BECLoMA: Augmenting stack traces with user review information}}
A ferramenta BECLoMA (\textit{Bug Extractor , Classifier and Linker of Mobile Apps}) é apresentada em \cite{Pelloni2018becloma}. Seu propósito é de vincular as avaliações de usuários aos processos de testes de aplicações, fornecendo aos desenvolvedores uma visão ampliada dos casos de testes e por consequência agilidade nas tomadas de ação. A ferramenta possui um \textit{crawler} capaz de minerar avaliações do \textit{Google Play Store} e já treinado para 6.600 avaliações. A parte central do software vincula os textos das avaliações aos códigos, retirando os métodos nativos e permitindo assim uma melhor visualização do problema.

A avaliação da ferramenta consistiu em responder à pergunta:

RQ1: Qual é a acurácia do BECLoMA em vincular as avaliações dos usuários aos relatórios de erro?

Foram realizados testes empíricos com 8 aplicações, número reduzido decorrente do esforço para geração de oráculos, construído por um desenvolvedor com 2 anos de experiência. A ferramenta apresentou precisão de 82\%, \textit{recall} de 75\% e \textit{F1-Score} de 78\%, confirmando que o processo \textit{Dice} para vinculação foi uma boa escolha, no entanto foi importante salientar que aproximadamente 18\% dos casos que deveriam ser vinculados a ferramenta não conseguiu associar.

A conclusão do estudo cita que a ferramenta pode no futuro ser evoluída e ter uma grande usabilidade para os desenvolvedores identificar erros e depurar códigos, e como próximo passo foi citada a utilização da ferramenta em conjunto com a IDE \textit{Android Studio}.

\subsection{\textit{Why People Hate Your App — Making Sense of User Feedback in a Mobile App Store}}

A ferramenta WisCom é apresentada em \cite{Fu2013whypeoplehate} como sendo uma alternativa para sumarizações de avaliações de usuários, podendo auxiliar tanto grandes empresas de software como os pequenos desenvolvedores. No estudo foram utilizados mais de 13 milhões de avaliações da \textit{Google Play Store}, em mais de 171 mil aplicações.

No estudo a ferramenta identificou inconsistências entre avaliações e classificações de usuários, os principais motivos pelos quais os usuários não gostam de uma aplicação, mudanças ao longo do tempo das avaliações e, por fim, tendências globais no mercado.

Permitindo a realização de \textit{drill down} de avaliações em 3 níveis distintos, através da ferramenta é possível identificar casos de inconsistências e razões pelo qual o usuário gosta ou não da aplicação, e fornecer visões intuitivas da loja de aplicativos que permitam identificar preferências dos usuários com relação a diferentes aplicações.

A primeira parte do artigo apresenta a comparação dos textos das avaliações com as respectivas notas, sinalizando os casos discrepantes.
A segunda parte do artigo tem a intenção de sinalizar os casos discrepantes para o operador da loja de aplicativos, desenvolvedores e até mesmo o próprio usuário.

Como o estudo envolveu mais de 13MM de avaliações, foi possível realizar uma análise independente da aplicação. O estudo conclui também que os usuários tendem a ser mais tolerantes para jogos do que outros tipos de aplicações, considerando softwares pagos.

\subsection{\textit{Analyzing and automatically labelling the types of user issues that are raised in mobile app reviews}}
A possibilidade e importância da rotulação das avaliações é apresentada em \cite{Mcilroy2016analyzing}. O estudo rotulou (em 11 classificações distintas) mais de 600 mil textos de 12 mil aplicações. O artigo cita que com as avaliações os desenvolvedores podem entender melhor as preocupações dos usuários, os proprietários de lojas de aplicativos podem identificar softwares anômalos e os usuários podem comparar aplicações semelhantes e decidir quais utilizar. Foi abordado que a natureza não estruturada e informal das avaliações, se utilizando muitas vezes de gírias e abreviações, complica a rotulagem automática de tais avaliações.

Além disso, o artigo cita que algumas empresas se especializaram em prover dados estatísticos e comparativos detalhados sobre avaliações para seus clientes, porém muitas vezes não é um trabalho orientado ao software.

As perguntas respondidas pelo estudo foram:

RQ1 - Quantos avaliações contém múltiplos tipos de problemas apontados?

A resposta foi que 22\% (Apple) e 30\% (Android) das avaliações são \textit{multi-label}, tendo uma média de palavras um pouco maior (passando de 32 para 41).

RQ2 - Quão bem é possível categorizar as avaliações em mais de um tipo de categoria?

No estudo a precisão das análises foram de até 66\%, com recall de até 65\%, sendo que a proposta de estudo futuro é de melhorar o processo de predição.

RQ3 - A abordagem de multi categorias é funcional para os \textit{stakeholders}?

O artigo informa que sim, e para isso considera provas de conceito.


\subsection{\textit{Retrieving and Analyzing Mobile Apps Feature}}
No artigo \cite{Iacob2013retrieving} é proposto um processo denominado MARA (\textit{Mobile App Review Analyzer}) para auxiliar na identificação de avaliações que apresentem solicitações de funcionalidades. No processo, a ferramenta se utiliza de \textit{Latent Dirichlet Allocation} (modelo estatístico de Processamento de Linguagem Natural) para as informações.

Na investigação foi considerado que textos com sarcasmos, não estruturados, reduzidos, com erros de pontuação são algumas barreiras a serem ultrapassadas no entendimento das avaliações.

As conclusões do estudo mostram que 23\% das avaliações sugerem novas funcionalidades ou alteração em casos existentes, e que a maioria das solicitações dos usuários diz respeito ao suporte aprimorado das aplicações, atualizações mais frequentes, novos níveis para jogos e mais opções de personalização.

\subsection{\textit{Online Reviews as First Class Artifacts in Mobile App Development}}
No estudo \cite{Iacob2014online} é apresentado o impacto das avaliações dos usuários no processo de engenharia de software, com argumentos explicando que os modelos e técnicas clássicas podem não ser os mais adequados. Baseado em evidências, o protótipo MARA visa oferecer integração das avaliações com o processo de engenharia de software, extraindo as solicitações de recursos e relatórios de erros.

Este novo estudo da ferramenta MARA estendeu a análise para classificação de erros, com aprendizagem baseada em exemplo de 3.279 avaliações randômicas, tratadas manualmente.

\subsection{\textit{User Reviews Matter! Tracking Crowdsourced - Reviews to Support Evolution of Successful Apps}}
Uma nova abordagem, denominada CRISTAL, é apresentada em \cite{Palomba2015userreviews} com o propósito de rastrear as avaliações comprovando que desenvolvedores que implementam as solicitações dos usuários são recompensados em termos de notas. O processo se utiliza de mecanismos \textit{crowdsourcing} sobre as avaliações das lojas de aplicativos, tendo como conjunto de estudo um total 100 aplicações Android.

Neste estudo foi também avaliada a possibilidade de relacionamento dos \textit{commits} e \textit{issues} com as avaliações dos usuários, porém apenas em um conjunto restrito de 10 aplicações, obtendo indicador \textit{F-Measure} de 75\% (77\% de \textit{precision} e 73\%de \textit{recall}).

As questões abordadas no estudo foram:

RQa: Quão preciso é o CRISTAL na identificação de links entre avaliações informativas e \textit{issues} e \textit{commits}?

O artigo explica a possibilidade de associação entre as avaliações dos usuários e os respectivos \textit{commits} das aplicações, sendo esta associação com uma precisão de 77\% e recall de 73\%.

RQc: Até que ponto os desenvolvedores cumprem as avaliações ao trabalhar em um novo lançamento de aplicação?

Os resultados alcançados mostraram que, na média, os desenvolvedores implementam 49\% das solicitações nas avaliações enquanto trabalham no lançamento de nova aplicação.

RQe: Qual é o efeito de um mecanismo análise massiva de avaliações (para planejar e implementar mudanças futuras) no sucesso da aplicação?

Confirmado através de uma correlação de 0,59 entre a cobertura das avaliações e a alteração de nota entre a versão anterior e nova da aplicação. Este percentual foi validado informado que a análise qualitativa suporta parte dos resultados quantitativos encontrados.

Como ameaça do estudo citou-se que a nota da aplicação foi melhorada decorrente de outras formas, como por exemplo uma importante entrega (\textit{feature}) disponibilizada. No entanto, a intenção do artigo é de apresentar uma correlação quantitativa ao invés de uma relação direta de causa e efeito. Também foram encontradas evidências através de mineração dos dados e discussões entre os autores de casos em que o mesmo usuário melhorou a nota após a liberação de uma nova versão da aplicação.

A conclusão do artigo é que as notas das aplicações aumentam quando os desenvolvedores implementam alterações que visam responder às avaliações dos usuários.

\subsection{\textit{Crowdsourcing user reviews to support the evolution of mobile apps}}
Na publicação \cite{Palomba2018crowdsourcing} a abordagem CRISTAL foi complementada com uma pesquisa junto a 73 desenvolvedores.

Os resultados qualitativos da pesquisa corroboram as informações quantitativas encontradas através da abordagem. Os desenvolvedores concordam que as avaliações representam uma valiosa informação, levando a correções difíceis de serem identificadas nos testes, implementação de recursos de sucesso e requisitos não funcionais. A grande maioria dos entrevistados (90\%) acredita que estas implementações têm um efeito de sucesso da aplicação.

\subsection{\textit{Mobile App Evolution Analysis based on User Reviews}}
O artigo \cite{Li2018MobileAE} associa avaliações com versões disponibilizadas da aplicação (\textit{releases}), porém tendo apenas como objeto de estudo a aplicação \textit{Whatspapp}.

Ao longo do estudo determinou-se quais foram as principais versões da aplicação e então realizada uma análise de sentimento sobre as avaliações enviadas entre estas versões.

Após agrupamento das palavras chave, verificou-se o comportamento destes grupos ao longo dos períodos. A conclusão foi de que se trata de um estudo investigativo exploratório, propondo que estudos futuros abordem novas categorias de software, plataforma, e principalmente que um sistema possa ser desenvolvido com base neste método de forma a minerar as avaliações e prover dados contínuos aos desenvolvedores.

O processo consistiu em determinar quais foram as principais versões da aplicação e em seguida realizar uma análise de sentimento sobre as avaliações enviadas entre as publicações destas versões.

\subsection{\textit{Do Android App Users Care about Accessibility? An Analysis of User Reviews on the Google Play Store}}
O estudo \cite{ihc2019} se encontra melhor detalhado na seção \ref{chap:atividades}, e dá subsídios como um estudo piloto para esta proposta de pesquisa.

\subsection{\textit{Accessibility in User Reviews for Mobile Apps: An Automated Detection Approach}}
O estudo \cite{rochestertamjeed} aborda a investigação das avaliações utilizadas no artigo \cite{ihc2019}, tomando parte da análise manual como oráculo para o mecanismo de aprendizado de máquina proposto nessa nova pesquisa.

%No que diz respeito às notas em lojas de aplicativos, em \cite{Yan2019currentstatus} não foi identificado um relacionamento direto entre a nota e as falhas de acessibilidade observadas na pesquisa.

\section{Considerações Finais}
%faço um resumo, crio o meu entendimento crítico e dou um gancho para o meu trabalho de pesquisa
%vimos que falta acessibilidade, ela é importante, muita gente não implementa, tem um monte de problema...
%Os comentários impulsionam os desenvolvedores, mas não temos estudos envolvendo
Em decorrência de relevante parcela da população mundial possuir algum grau de deficiência, bem como do avanço tecnológico que permitiu um aumento considerável da utilização de software para dispositivos móveis, entende-se como fundamental a evolução das aplicações no que diz respeito à sua acessibilidade.

Apesar desta importância e mesmo em aplicações móveis populares, identifica-se problemas de acessibilidade conforme os padrões internacionalmente reconhecidos, o que leva ao indício de que não há a devida significância para este tema junto aos desenvolvedores, ou mesmo não há uma pressão de mercado que induza a esta priorização durante a elaboração de novas versões.

Considerando que as avaliações são uma importante fonte de retroalimentação para os desenvolvedores e dada a importância da acessibilidade no cenário mundial, entende-se que existe a necessidade de um estudo do relacionamento entre as avaliações em lojas de aplicativos, e as informações cadastradas pelos desenvolvedores para as solicitações de modificações e alterações, tendo em conta o tema acessibilidade.
