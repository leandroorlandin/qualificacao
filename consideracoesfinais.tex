Esta proposta de pesquisa visa apresentar um panorama até o momento não identificado em outro estudo, correlacionando as informações de avaliações de usuários com as modificações dos desenvolvedores de aplicações móveis com foco no tema acessibilidade.

Considerando o cenário dos dados base das avaliações, solicitações de modificações e alterações já se encontrarem levantados conforme Seção~\ref{sec:atividadesrealizadas} bem como o prêmio do estudo piloto citado na Seção~\ref{sec:publicacoes}, entende-se que esta proposta é factível de acordo com o cronograma apresentado na Seção~\ref{sec:cronograma}.

A importância desta pesquisa se justifica pela contribuição aos estudos relacionados à acessibilidade digital e principalmente no incentivo de futuras pesquisas e ações voltadas à conscientização e treinamento de desenvolvedores no que diz respeito à acessibilidade.

Os resultados desta pesquisa podem contribuir com os estudos relacionados à acessibilidade digital em aplicações móveis, fomentando pesquisas e ações voltadas à conscientização e treinamento de desenvolvedores, e igualmente a criação de novos recursos para
a implementação e avaliação de acessibilidade digital.