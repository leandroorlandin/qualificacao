As demandas realizadas por usuários em suas avaliações publicadas em lojas de aplicativos impactam as ações dos desenvolvedores e das empresas na correção de defeitos e no planejamento de novas funcionalidades que serão disponibilizadas em novas versões. Esta proposta de pesquisa visa investigar um cenário até o momento não explorado por outros estudos, que são os aspectos de acessibilidade abordados nas avaliações e correlação deste tipo de demanda com as solicitações de modificações e alterações de fato realizadas nas aplicações móveis para que sejam mais acessíveis.

Até o momento, foi realizado um estudo piloto com aproximadamente 700 aplicações e que forneceram evidências suficientes para motivar um estudo mais amplo e detalhado. Para este novo estudo, já foram selecionados 900 aplicações e extraídas 1,5 milhão de avaliações, 41 mil solicitações de mudanças e mais de 1 milhão de alterações realizadas no código das respectivas aplicações investigadas. 

Os resultados desta pesquisa vão mostrar se os usuários que possuem alguma deficiência ou que encontram barreiras de acessibilidade estão utilizando esta importante ferramenta que são as avaliações publicadas nas lojas de aplicativos para fazer suas demandas por melhorias da acessibilidade das aplicações. O estudo inicial realizado mostrou que as demandas deste tipo são mínimas, mas um estudo ampliado com mais avaliações coletada e com novas palavras-chave para seleção de aplicativos será importante para dar mais robustez aos resultados. 

A confirmação de que as avaliações são pouco exploradas para demandar melhorias na acessibilidade das aplicações pode ser utilizada como subsídio de ações cujo objetivo seja promover o maior engajamento de usuários com deficiência na exigência de aplicativos com maior qualidade e com mais acessibilidade uma vez que os comentários dos usuários tem tido impacto nas ações dos desenvolvedores no que se refere à evolução de suas aplicações móveis. 


%A importância desta pesquisa se justifica pela contribuição aos estudos relacionados à acessibilidade digital e principalmente no incentivo de futuras pesquisas e ações voltadas à conscientização e treinamento de desenvolvedores no que diz respeito à acessibilidade.

%Os resultados desta pesquisa podem contribuir com os estudos relacionados à acessibilidade digital em aplicações móveis, fomentando pesquisas e ações voltadas à conscientização e treinamento de desenvolvedores, e igualmente a criação de novos recursos para
%a implementação e avaliação de acessibilidade digital.
