

\section{Contextualização}

Uma importante parcela da população mundial possui algum tipo de deficiência. 
De acordo com a Organização Mundial da Saúde - OMS (\textit{World Health Organization - WHO})\footnote{http://documents.worldbank.org/curated/en/665131468331271288/Main-report}, 
as estimativas são de que este público seja superior a 1 bilhão de pessoas. Considerando a população brasileira, as estimativas do censo de 2010\footnote{https://biblioteca.ibge.gov.br/visualizacao/periodicos/94/cd\_2010\_religiao\_deficiencia.pdf} são de que aproximadamente 18,8\% apresenta algum tipo de deficiência visual, 5,1\% deficiência auditiva, 7\% deficiência motora e 1,4\% deficiência mental ou intelectual. É importante salientar também que esses percentuais se agravam para a população idosa.

As pessoas com deficiência enfrentam barreiras na execução das mais diversas atividades cotidianas, incluindo o uso de serviços e de produtos eletrônicos. 
Neste contexto, a acessibilidade digital tem o objetivo de remover as barreiras que possam impedir ou dificultar os usuários a perceber, entender e operar produtos digitais de forma completa, segura e autônoma~\cite{wcag,w3cwai}.
Embora a acessibilidade digital tenha impacto na usabilidade e na qualidade global de qualquer software~\cite{Gay2018,ISO25010}, ela tem o foco nas necessidades específicas de pessoas com deficiência~\cite{ISO9241:11}.


Muitos avanços foram realizados nesta área ao longo dos anos para promover a acessibilidade digital: a criação de tecnologia assistiva, como leitores de tela e navegação por voz; 
a criação de padrões e guias de acessibilidade para orientar a produção de conteúdo e aplicações acessíveis, como o WCAG (\textit{Web Content Accessibility Guideline}) do W3C (\textit{The World Wide Web Consortium});
a criação de ferramentas para o desenvolvimento e avaliação de acessibilidade~\cite{Silva2018survey};
e a publicação de diversas leis que tornam obrigatório o desenvolvimento de produtos digitais acessíveis na esfera pública e privada~\cite{Lazar2019}.
A acessibilidade digital teve o foco inicial nas páginas e aplicações \textit{web} dado que diversas organizações adotaram esta plataforma para a divulgação de informação e para a realização de negócios, mas o aumento do uso de \textit{smartphones} e de suas aplicações trouxe também o foco para a acessibilidade digital móvel.


%Paralelo a este cenário, identifica-se também um aumento considerável do número de softwares para dispositivos móveis\footnote{https://www.statista.com/statistics/266210/number-of-available-applications-in-the-google-play-store/} denominados como aplicações móveis (ou apenas \textit{apps}). Esta explosão de \textit{apps} disponíveis para serem instalados e utilizados em celulares, tablets ou até mesmo em computadores, passou de 30 mil em 2009 para mais de 3 milhões em 2018, evidenciando assim a importância dos mesmos em muitos aspectos de nossas atividades cotidianas \cite{storeanalysis}.

%A partir dos fatores acima, índices de deficiência da população e utilização dos dispositivos móveis no cotidiano, atesta-se a importância da acessibilidade digital neste novo panorama mundial, essa representando a capacidade do software de ser utilizado independentemente da condição física, mental ou intelectual do usuário \cite{w3cwai}. 


\section{Lacuna de Pesquisa}

Infelizmente, 
apesar da existência de recursos para o desenvolvimento de aplicações acessíveis, 
diversos estudos evidenciaram uma falta geral de acessibilidade em aplicações móveis de diversas categorias, tamanhos, complexidade e popularidade~\cite{serra2015accessibility,eler2018mate,Yan2019currentstatus,Vendome2019,Alshayban2020,AcostaVargas2020}.
O mesmo fenômeno tem sido observado em aplicações \textit{web}, e por isso diversos estudos foram conduzidos para entender as razões que explicam a falta de acessibilidade observada. A maioria desses estudos investigou a consciência, o conhecimento e as questões organizacionais que impactam o desenvolvimento de software acessível e constatou que a maioria dos desenvolvedores tem pouco conhecimento sobre acessibilidade, as demandas por manutenção e novos produtos de software são muito altas, os prazos são curtos, não há treinamento adequado e o tema acessibilidade não é uma prioridade das organizações~\cite{  
lazar2004improving,Freire2008survey,oliveira2017strategies,Inal2019,barzilai2008factors,Putnam:2012}. Um estudo recente realizado com mais de 870 desenvolvedores de aplicações móveis no Brasil mostrou um comportamento semelhante nesta plataforma específica.


Se por um lado o conhecimento dos desenvolvedores e o contexto organizacional não favorecem o desenvolvimento de aplicações acessíveis, um outro fator pode fazer a diferença neste cenário: as requisições dos usuários. 
No contexto do desenvolvimento e evolução de aplicações móveis, 
é comum que as organizações utilizem avaliações (notas e comentários) realizadas pelos usuários nas lojas de aplicativos para planejar a correção de defeitos e o lançamento de novas funcionalidades, pois isso geralmente resulta em melhores avaliações, o que pode dar destaque e potencializar a instalação por outros usuários~\cite{nayebi,Palomba2015userreviews,Palomba2018crowdsourcing,Li2018MobileAE,Ciurumelea2017analyzing,Ortega2015thesis}.

Considerando que os comentários e as notas recebidas nas avaliações dos usuários são utilizados para corrigir falhas e evoluir aplicações móveis, duas questões de pesquisa são destacadas neste trabalho:
\begin{itemize}
 \item Os usuários de aplicações móveis mencionam aspectos relacionados à acessibilidade em suas avaliações publicadas nas lojas de aplicativos?
 \item Os problemas de acessibilidade relatados nas avaliações dos usuários têm impacto na evolução das aplicações móveis?
\end{itemize}

Diversos estudos sobre avaliações dos usuários em lojas de aplicativos foram realizados para entender as demandas dos usuários e as respostas dos desenvolvedores e das organizações~\cite{Iacob2013retrieving,Pagano2013userfeedback,Iacob2014online,Mcilroy2016analyzing,Sorbo2017surf,Ciurumelea2017analyzing,Ortega2015thesis,Li2018MobileAE,Pelloni2018becloma,Panichella2015how}. Os trabalhos encontrados na literatura classificam as avaliações dos usuários de diferentes formas: 
relato de falha, reclamações em geral, requisição de nova funcionalidade, solução de propostas, problemas com o custo, incompatibilidades, dificuldades com a rede, preocupações com segurança e privacidade, consumo de memória e bateria, elogios, e questões relacionadas à interface com o usuário.
Apesar de alguns estudos classificarem avaliações relacionadas à interface com o usuário, nenhuma delas fez uma diferenciação ou um detalhamento sobre os aspectos de acessibilidade abordados pelos usuários, tampouco enfatizaram a evolução da acessibilidade das aplicações. 


\section{Objetivo} 

O objetivo geral deste projeto de pesquisa é investigar se as avaliações feitas por usuários abordam aspectos relacionados à acessibilidade do produto, e se essas avaliações têm algum efeito na melhoria da acessibilidade da aplicação. Mais especificamente, deseja-se saber:
\begin{itemize}
 \item A quantidade e a distribuição de avaliações relacionadas à acessibilidade da aplicação
 \item A diversidade de tipos de violações de acessibilidade mencionadas nas avaliações
 \item O impacto das questões de acessibilidade nas notas atribuídas pelos usuários
 \item A quantidade e a distribuição de modificações relacionadas à acessibilidade da aplicação realizadas no código-fonte
 %\item A evolução da quantidade de violações encontradas nas diferentes versões da aplicação
\end{itemize}



\section{Justificativa e relavância}

Como as avaliações dos usuários são uma importante ferramenta para direcionar a evolução e a melhoria de aplicações móveis, 
é importante entender se e como as avaliações publicadas em lojas de aplicativos estão sendo utilizadas para informar aos desenvolvedores e às organizações questões relacionadas à acessibilidade das aplicações avaliadas. 
Além disso, também é fundamental entender se as questões de acessibilidade abordadas pelos usuários são levadas em consideração na evolução das aplicações avaliadas. 

Desta forma, os resultados da investigação aqui proposta podem dar evidências de que as pessoas com deficiência deveriam manifestar explicitamente as barreiras que enfrentam para usar aplicações, e assim pressionar desenvolvedores e organizações a aperfeiçoarem seus produtos.
Adicionalmente, mostrar que os diversos tipos de violações de acessibilidade trazem problemas reais para usuários e podem motivar desenvolvedores a levar em consideração as recomendações dos padrões e guias de acessibilidade enquanto projetam as interfaces de suas aplicações.


\section{Metodologia}

%Desta forma, esta proposta de pesquisa visa realizar um estudo amplo e generalizado envolvendo o tema acessibilidade sobre os comentários dos usuários na loja de aplicativos móveis \textit{Google Play Store}, \textit{commits} e \textit{issues} cadastrados pelos desenvolvedores em repositório aberto de código fonte \textit{Github}.

%As justificativas que visam embasar esta investigação pretendem 
%i) identificar se existe um volume significativo de comentários envolvendo acessibilidade e sua distribuição de acordo com as categorias dos \textit{apps},
%ii) analisar a diversificação dos comentários de acordo com padrões internacionais de acessibilidade \cite{bbc} e
%iii) se existe um relacionamento entre a classificação do comentário de acordo com a acessibilidade e à respectiva nota dada ao aplicativo pelo próprio usuário.
%iv) a correlação quantitativa entre os comentários, commits e issues

%Com este estudo, será possível confirmar a existência de uma pressão dos usuários frente aos desenvolvedores e empresas de aplicativos no que diz respeito às funcionalidades envolvendo acessibilidade.

Os objetivos desta proposta de pesquisa serão alcançados por meio da execução das seguintes atividades: 
\begin{itemize}
 \item Seleção de um conjunto de aplicações móveis cujas avaliações publicadas na loja de aplicativos serão analisadas.
 \item Obtenção das avaliações (comentários e notas atribuídas) dos usuários das aplicações selecionadas.
 \item Obtenção das sugestões de modificações (correção de defeitos, melhorias, novas funcionalidades - \textit{issues}) e alterações (\textit{commits}) no código das aplicações selecionadas por meio do acesso a seus repositórios de código.
 \item Seleção das avaliações dos usuários, sugestões de modificações e alterações que abordam aspectos da acessibilidade das aplicações selecionadas.
 \item Análise das avaliações, sugestões de modificações e alterações relacionadas à acessibilidade das aplicações.
 \end{itemize}

A metodologia deste trabalho está detalhada no Capítulo~\ref{chap:proposta}, onde todas as decisões tomadas para a condução deste projeto de pesquisa são apresentadas e justificadas. 

%Será definido um conjunto de aplicações móveis, com a condição inicial de que possuam versões de suas instalações (arquivos de instalação com extensão \textit{apk}) disponíveis no repositório F-Droid\footnote{https://f-droid.org/}. Este requisito permitirá que a pesquisa seja posteriormente evoluída, através de comparações entre os problemas de acessibilidade encontrados em diferentes versões do software, fora do escopo deste projeto. É importante que o levantamento selecione aplicações de diferentes categorias, como por exemplo: navegação, finanças e leitura.

%Uma nova ferramenta, em linguagem Python, será implementada para coletar os comentários dos aplicativos e validar a existência de versões do software de instalação no repositório de F-Droid. Esta condição inicial de validação poderá ser excluída se por ventura o conjunto for restrito a determinadas categorias ou mesmo possuam reduzido número de comentários envolvendo acessibilidade.

%Com base no \textit{W3C} \cite{wcag} e no \textit{BBC Guidelines} \cite{bbc}, serão estabelecidas palavras-chave que, em existindo no comentário, indicarão a possibilidade do mesmo se referir a uma diretriz de acessibilidade.

%Em posse dos comentários e das palavras-chave, será possível iniciar um processo de averiguação manual se os mesmos abordam o tema acessibilidade e qual foi a mensagem transmitida pelo usuário (pedido de correção ou cumprimentos pelo aplicativo).

%A mesma metodologia aplicada para a obtenção e avaliação dos comentários será utilizada sobre os textos informativos dos desenvolvedores quando da realização dos \textit{commits} e cadastramento das \textit{issues} no repositório de código fonte aberto \textit{Github}.

%Em posse destes dados, será realizado então um estudo que visa identificar uma correlação quantitativa entre as 3 bases de informações: comentários, \textit{commits} e \textit{issues}.


%Com a intenção de endereçar este tema auxiliando os designers e desenvolvedores de software a criar produtos acessíveis, vários padrões têm sido propostos e amplamente utilizados, principalmente aqueles ligados ao \textit{World Wide Web Consortium - W3C} \cite{wcag} e ao \textit{BBC Guidelines} \cite{bbc}. No âmbito nacional destaca-se o Modelo de Acessibilidade em Governo Eletrônico, ou simplesmente eMAG \cite{emag}. Todavia, é possível observar em estudos que ainda existe uma falta de acessibilidade nas aplicações \cite{smartcities,Eler2018mate,Quispe2018accessibility,Serra2015accessibility,Yan2019currentstatus}.

%É fundamental que não apenas os requisitos funcionais sejam garantidos pelas aplicações, mas também a disponibilização das mesmas considerando sua acessibilidade \cite{Oliveira2017strategies}. No âmbito de \textit{apps}, tem-se como importante característica suas constantes atualizações através de novas versões nas lojas de aplicativos (e.g. \textit{Google Play Store}) \cite{nayebi,Palompa2018crowdsourcing}. Este processo de frequente revisão ao longo do tempo torna as aplicações mais maduras, com inclusão de novas funcionalidades e correções de erros.

%Em muitos estudos identifica-se que tanto as classificações quanto os comentários dos usuários nas lojas de aplicativos são um importante direcionador para os desenvolvedores no momento do lançamento de novas versões \cite{Ciurumelea2017analyzing,Li2018MobileAE,Ortega2015thesis,Palomba2015userreviews,Palompa2018crowdsourcing,Pelloni2018becloma}. Considerando assim que os comentários orientam o desenvolvimento das \textit{apps} e que mesmo as mais maduras e populares apresentam problemas de acessibilidade \cite{Eler2018mate}, as dúvidas que surgem é se os usuários reportam suas dificuldades referentes à acessibilidade nas lojas de aplicativos, e do quão relevante são estes comentários para os desenvolvedores.



%Diferentes taxonomias são empregadas nos estudos dos comentários \cite{Ciurumelea2017analyzing,Sorbo2017surf,Iacob2013retrieving,Iacob2014online,Li2018MobileAE,Mcilroy2016analyzing,Ortega2015thesis,Pagano2013userfeedback,Panichella2015how,Pelloni2018becloma}, e apesar de alguns destes trabalhos citarem inclusive interfaces com usuário, não se observa estudos que possuam detalhamento a respeito dos problemas relacionados à acessibilidade digital.





%falar bem rapidamente como chegaremos lá
%selecionaremos um conjunto de aplicações
%implementar ferramentas para coletar estes comentários
%analisaremos estes comentários
%definiremos palavras-chave
%validação manual
%utilizaremos PLN (mas precisamos definir para quê utilizar a mesma) se dividirmos quantos tópicos a pessoa fala em cada comentários seria interessante
%quando a pessoa apenas reclama de algo a nota é ruim, porém quando existem outros assuntos envolvidos e algum deles pode ser elogios, a nota pode ser melhor.....
%análise de sentimento para extrair algo



\section{Organização}

Esta proposta de pesquisa está organizada da seguinte forma:
no Capítulo~\ref{chap:background} são apresentados os conceitos fundamentais que embasam este projeto de pesquisa,
no Capítulo~\ref{chap:proposta} a metodologia deste projeto de pesquisa é apresentada em detalhes,
e no Capítulo~\ref{chap:atividades} as atividades já realizadas desta proposta de pesquisa são apresentadas.
Por fim, algumas considerações finais são apresentadas no Capítulo~\ref{chap:conclusao}.


